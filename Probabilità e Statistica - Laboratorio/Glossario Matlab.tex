\documentclass{article}
\usepackage{amsmath}
\usepackage{bbding}
\usepackage{pifont}
\usepackage{booktabs}
\usepackage{amssymb}
\usepackage[margin=0.95in]{geometry}
\usepackage{listings}
\begin{document}

\title{Formulario Matlab\\Probabilità e Statistica}
\author{Chiara Solito - Bioinformatica}
\date{II Semestre - A.A. 2020/2021}
\maketitle
\section{Basi}
\begin{itemize}
    \item $\%$
    \subitem Non compila e non esegue il testo (commenti)
    \item Mostrare qualcosa a display (output)
    \subitem \texttt{disp()}
    \item Vettore
    \subitem \texttt{a1 = [ 2 5 1 ]; $\%$\textit{riga}}
    \item Grafico a Torta
    \subitem \texttt{a1 = [ 2; 5; 1 ]; $\%$\textit{colonna}}
    \item Appendere i vettori
    \subitem \texttt{u = [r,w]}
    \item Operatore colonna
    \subitem Genera vettori spaziati regolarmente
    \subitem \texttt{u = m:q:n}
    \item Operatore linspace
    \subitem Genera vettori riga spaziati linearmente 
    \subitem \texttt{linspace(x1,x2,n)}
    \item Matrice
    \item \texttt{a1 = [ a b c; d e f; g h i]}\\
    \\
    $\begin{bmatrix}
        a & b & c\\
        d & e & f\\
        g & h & i
    \end{bmatrix}$
    \item Accedere alle matrici
    \subitem \texttt{A(:,3)} denotes all the elements in the 3rd column of the matrix A
    \subitem \texttt{A(:,2:5)} denotes all the elements in the 2nd through 5th columns of A
    \subitem \texttt{A(2:3,1:3)} denotes all the elements in the 2nd and 3rd rows that are also in the 1st through 3rd columns
    \subitem \texttt{v = A(:)} creates a vector v consisting of all the columns of A stacked from first to last
    \subitem \texttt{A(end,:)} denotes the last row in A
    \subitem \texttt{A(:,end)} denotes the last column in A
    \item Stringa
    \subitem \texttt{month = 'August'}
    \item Cella 
    \subitem \texttt{a = cell(3,3)}
    \item Struttura
    \subitem \texttt{student.year = 2; \%come il c}
\end{itemize}

\section{Funzioni per matrici}
\begin{itemize}
    \item Concatenare matrici
    \subitem \texttt{B = [A A+32; A+48 A+16]}
    \item Creare matrici dai vettori
    \subitem \texttt{c = [a b];} o \texttt{c = [a; b];}
    \item Cancellare righe e colonne
    \subitem \texttt{X = magic(4)} e \texttt{X(:,3)=[]}
    \item Creare matrici di zeri
    \subitem \texttt{C = zeros(2,2)}
    \item Creare matrici di 1
    \subitem \texttt{C = ones()}
    \item Creare matrici di numeri random
    \subitem \texttt{C = rand()}
    \item Trovare un elemento diverso da zero
    \subitem \texttt{[k,j] = find(A)}
    \item Trova la lunghezza (degli array) o il massimo tra altezza e lunghezza di una matrice
    \subitem \texttt{length(A)}
    \item Trova il massimo
    \subitem \texttt{max(A)}
    \item Trova il minimo
    \subitem \texttt{min(A)}
    \item Trova la taglia
    \subitem \texttt{size(A)}
    \item Somma gli elementi di ogni colonna
    \subitem \texttt{sum(A)}
\end{itemize}

\section{Operaioni per array e matrici}
\begin{itemize}
    \item Addizione
    \subitem \texttt{+}
    \item Sottrazione
    \subitem \texttt{-}
    \item Moltiplicazione elemento per elemento
    \subitem \texttt{.*}
    \item Divisione elemento per elemento
    \subitem \texttt{./}
    \item Divisione sinistra elemento per elemento
    %\subitem \texttt{.\textbackslash}
    \item Potenza elemento per elemento
    \subitem \texttt{.\^}
    \item Trasposta non coniugata di un array
    \subitem \texttt{.\'}
\end{itemize}

\section{Grafi e Tavole di Frequenza}
\begin{itemize}
    \item Grafico a Bastoncini
    \subitem \texttt{stem()}
    \item Grafico a Linee
    \subitem \texttt{plot()}
    \item Grafico a Barre
    \subitem \texttt{bar()}
    \item Grafico a Barre Orizzontali
    \subitem \texttt{barh()}
    \item Grafico a Torta
    \subitem \texttt{pie()} e \texttt{pie3()}
    \item Boxplot
    \subitem \texttt{boxplot()}
    \item Sublplot?
    \subitem \texttt{subplot(A)}
\end{itemize}

\section{Basic Plotting}
\texttt{x = 0:pi/100:2*pi;\\
y = sin(x);\\
plot(x,y)\\
xlabel('x = 0:2 %\textbackslash pi')\\
ylabel('sin(x)')\\
title('Plot of the Sine Function')}

\section{Statistica Descrittiva - Funzioni}
\begin{itemize}
    \item Frequenza
    \subitem \texttt{hist()}
    \item Tabella delle frequenze
    \subitem \texttt{tabulate()}
    \item Media
    \subitem \texttt{mean() \%= sum(A)/n}
    \item Mediana
    \subitem {median() \%50th percentile}
    \item Moda
    \subitem \texttt{mode() \%most frequent value}
    \item Deviazione Standard
    \subitem \texttt{std()}
    \item Varianza
    \subitem \texttt{var()}
    \item Skewness
    \subitem \texttt{skewness()}
    \item Kurtosi
    \subitem \texttt{kurtosis}
    \item (\texttt{nanmean() \%Mean, ignoring NaN values})
    \item Range
    \subitem \texttt{range()}
    \item Massimo
    \subitem \texttt{max()}
    \item Minimo
    \subitem \texttt{min()}
    \item Interquartile range
    \subitem \texttt{iqr()}
    \item Percentile del data set
    \subitem \texttt{prctile()}
    \item Quantile del data set
    \subitem \texttt{quantile()}
    \item Coefficiente di correlazione
    \subitem \texttt{corrcoef()}
    \item Grafico di Dispersione per gruppo
    \subsubitem \texttt{gscatter()}
    \item Istogramma bivariato
    \subsubitem \texttt{hist3()}
    \item Grafico di Dispersione con istogrammi marginali
    \subsubitem \texttt{scatterhist()}
\end{itemize}

\section{I Cicli}
\begin{itemize}
    \item FOR
    \subitem \texttt{for loopvar = range}
    \subsubitem \texttt{action}
    \subitem \texttt{end}
\end{itemize}

\section{Elementi di Probabilità}
\begin{itemize}
    \item Permutazione
    \subitem \texttt{perms()}
    \item Coefficiente Binomiale
    \subsubitem \texttt{nchoosek()}
\end{itemize}

\section{Distribuzioni di Probabilità}
\begin{itemize}
    \item Funzione di densità di probabilità
    \subitem \texttt{pdf()}
    \item Funzione di distribuzione cumulativa
    \subsubitem \texttt{cdf()}
    \item Inversa di cdf
    \subitem \texttt{inv()}
    \item Generazione di numeri casuali
    \subsubitem \texttt{rnd()}
    \item Generazione di numeri casuali uniformemente distribuito sull'intervallo (0,1)
    \subsubitem \texttt{unifrnd()}
    \item Media e Varianza
    \subitem \texttt{stat()}
    \item Testing di ipotesi
    \subsubitem \texttt{test()}
    \item Funzione di errore
    \subsubitem \texttt{erf()}
    \item Stima di Massima Verosimiglianza
    \subitem \texttt{erf()phat = mle(x,’distribution’,’dist’)}
    \subitem returns the maximum likelihood estimate of the
    parameters that is assumed to be originated from
    the specified distribution by ‘dist’.
    \subitem \texttt{phat = mle(means)}
    \subitem Distribution of means from repeated random
    samples of an exponential distribution
\end{itemize}

\begin{center}
    \begin{tabular}{c|c|c}
        \toprule 
        Distribuzione & PDF & CDF\\
        \midrule
        Normale & normpdf() & normcdf()\\
        Uniforme (cont) & unifpdf() & unifcdf() \\
        Essponenziale & exppdf() & expcdf()\\
        Chi-square & chi2pdf() & chi2cdf() \\
        F & fpdf() & fcdf()\\
        t & tpdf() & tcdf()\\
        Binomiale & binopdf() & binocdf()\\
        Poisson & poisspdf() & poisscdf()\\
        Ipergeometrica & hygepdf() & hygecdf()\\
        \bottomrule
    \end{tabular}   
\end{center}

\section{Esercizi}
Costruire un intervallo di confidenza al 95\% per $\mu$, l’intensità effettiva. Determinare, per
gli stessi dati, un intervallo di confidenza al 99\% per $\mu$.
\begin{lstlisting}[language=Matlab]
    data = [17 21 20 18 19 22 20 21 26 29];
    N = length(data);
    sigma = 3;
    alpha1 = 0.05;
    z1 = norminv(1-alpha1/2);
    x_bar = mean(data);

    %calcolo del primo intervallo di confidenza
    ConfidenceInterval95 = [x_bar - z1*sigma/sqrt(N) , x_bar + z1*sigma/sqrt(N)];
    %[17.44 21.16]

    %calcolo del secondo intervallo di confidenza
    alpha2 = 0.01;
    z2 = norminv(1-alpha2/2);
    ConfidenceInterval99 = [x_bar -z2*sigma/sqrt(N), x_bar + z2*sigma/sqrt(N)];
    %[16.86 21.74]
\end{lstlisting}

Si suppone che l'intensità del segnale sia uguale a 20. Verificare se questa ipotesi è plausibile. Usare una significativa del 5\%,
\begin{lstlisting}[language=Matlab]
    x_true = 20;
    [h,p] = ztest(data, x_true, sigma); 
    % h = 0 lo ztest non rifiuta l'ipotesi nulla 
    %(H0: mu=20) ad un lvl del 5%
\end{lstlisting}

Costruire un modello (fit) di tipo lineare che descriva l'andamento degli incassi al botteghino in funzione del numero di weeken
\begin{lstlisting}[language=Matlab]
    %linear fit: least squares regression methos
    b = polyfit(x,y,1);
\end{lstlisting}

Costruire una prima figura che rappresenti un diagrama a dispersione dei dati
con sovrapposto il modello di tipo lineare (assegnare correttamente titolo e label alla figura)

\begin{lstlisting}[language=Matlab]
    %diagramma di dispersone dei dati e visualizzazione del modello lineare
    figure(1)
    scatter(x,y); xlabel('weekend'); ylabel('incassi')
    hold on
    plot(x, x*b(1)+b(2))
    title('Diagramma di dispersione dei dati e retta di regressione (modello lineare y = beta_1 x + beta_2'))
    hold off
\end{lstlisting}

Determinare se il modello lineare è appropriato a) calcolando e visualizzando in una seconda figura i residui (assegnare correttamente titolo e label),
b) calcolando il coefficiente di determinazione ($R^2$).

\begin{lstlisting}[language=Matlab]
    %calcolo dei residui e visualizzazione
    res = polyval(b,x)-y;
    figure(2)
    stem(res)
    title('Residui')
    xlabel(1weekend'); ylabel('residui')

    %calcolo del coefficiente di determinazione
    SSR = sum(res.^2); %residual sum of square
    SYY = sum((y-mean(y)).^2); %devianza: total sum of squares
    R2 = 1 -SSR/SYY; %coefficiente di determinazione
\end{lstlisting}

Calcolare la probabilità che al massimo 3 persone selezionate abbiano un determinato disturbo. P((X $\leq$ 3))

\begin{lstlisting}[language=Matlab]
    p = 0.2; %probabilita' dello 0.2 che una persona soffra di una malattia
    n = 25; %seleziono 25 persone
    cdf3 = binocdf(3,n,p); %P(X<=3)
\end{lstlisting}

Ottenere i valori della funzione di massa di probabilità (probability mass funztion) per n=6,
p = 0.3, e successibamente per n = 6, p = 0.7 e visualizzare i risultati nella stessa figura (due subplot) usando la funzione bar. Assegnare correttamente titolo e label a ciascun subplot.
\textit{Nota: quando chiede la probability mass function lei dà come risposta la binopdf, la pdf binomiale}
\begin{lstlisting}[language=Matlab]
    x = 0:1:25;

    n1 = 6;
    p1 = 0.3;
    pmf1 = binopdf(x, n1, p1); % pmf binomiale con parametri (n1,p1)

    n2 = 6;
    p2 = 0.7;
    pmf2 = binopdf(x, n2, p2); %pmf binomiale con parametri (n2,p2)

    figure(1);
    subplot(2,1,1);
    bar(pmf1);
    title ('pmf binomiale con parametri (n = 6, p = 0.3)')
    xlabel('x');
    ylabel('P(X=x)');

    figure(1);
    subplot(2,1,2);
    bar(pmf2);
    title ('pmf binomiale con parametri (n = 6, p = 0.7)')
    xlabel('x');
    ylabel('P(X=x)');
\end{lstlisting}
\end{document}